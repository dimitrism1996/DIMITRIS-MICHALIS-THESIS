\bibitem{MAEA}
Y.S. Ong, P.B. Nair, and A.J. Keane. Evolutionary 
optimization of computationally expensive problems via 
surrogate modeling. \textit{AIAA Journal}, 
41(4): 687–696, 2003.

\bibitem{preprint_SMT}
M.A. Bouhlel, J.T. Hwang, N. Bartoli, R. Lafage, 
J. Morlier, and J.R.R.A. Martins. A Python surrogate 
modeling framework with derivatives. \textit{Advances 
in Engineering Software}, 135, 2019.

\bibitem{DOE}
D.C. Montgomery. Design and Analysis of Experiments. John 
Wiley \& Sons, New York, USA, 6th edition, 2005.

\bibitem{EAs} Κ. Χ. Γιαννάκογλου. Μέθοδοι Βελτιστοποίησης 
στην Αεροδυναμική. Πανεπιστημιακές Εκδόσεις Ε.Μ.Π., 
Αθήνα, σελίδες 125-130, 2006.

\bibitem{NSGA}
N. Srinivas and K. Deb. Multiobjective optimization using 
nondominated sorting in genetic algorithms. 
\textit{Evolutionary Computation}, 2(3): 221–248, 1995.

\bibitem{SPEA}
E. Zitzler and L. Thiele. Multiobjective evolutionary 
algorithms: A comparative case study and the Strength 
Pareto approach. \textit{IEEE Transactions on Evolutionary
Computation}, 3(4): 257–271, November 1999.

\bibitem{NSGA_2}
K. Deb, S. Agrawal, A. Pratap, and T. Meyarivan. A fast 
elitist non-dominated sorting genetic algorithm for 
multi-objective optimization: NSGA-II. In
\textit{Parallel Problem Solving from Nature – PPSN 
VI}, Paris, France, 2000. 

\bibitem{SPEA_2}
E. Zitzler, M. Laumans, and L. Thiele. SPEA2: Improving 
the strength Pareto evolutionary algorithm for 
multiobjective optimization. In \textit{Eurogen 2001, 
Evolutionary Methods for Design, Optimisation and Control 
with Applications to Industrial Problems}, Barcelona, 
pp. 19-26, 2002.

%--------------------Off-line---------------------
\bibitem{global_metamodel}
M. Farina. A neural network based generalized response 
surface multiobjective evolutionary algorithm. In 
\textit{2002 Congress on Evolutionary Computation – CEC 
’02}, Honolulu, HI, USA, May 2002.

\bibitem{DOE1}
J.R. Wagner, E.M. Mount, and H.F. Giles. \textit{Plastics 
Design Library, Volume: Extrusion}, chapter Design of 
Experiments, Elsevier Science, 2nd edition, pp. 291-308, 
2014.

\bibitem{DOE2}
A. Sethuramiah and R. Kumar. \textit{Modeling of Chemical 
Wear}, chapter Statistics and Experimental Design in 
Perspective, Elsevier Science, pages 129-159, 2016.

\bibitem{Random}
R. Mead, S. Gilmour, and  A. Mead. \textit{Statistical 
principles for the design of experiments}. Cambridge 
University Press, 1st ed., pp. 233-271, 2012.

\bibitem{Latin Hypercube}
M. Stein. Large Sample Properties of Simulations Using Latin 
Hypercube Sampling. \textit{Technometrics}, 29: 143-151, 
1987. 

\bibitem{LHS}
I. Ronald. Latin Hypercube Sampling. \textit{ResearchGate}, 
January 1999.

\bibitem{LHS method}
M. McKay, R. Beckman, and W. Conover. A comparison of three methods 
for selecting values of input variables in the analysis of output 
from a computer code. \textit{Technometrics}, 21: 239–245, 1979.

\bibitem{maximin}
M. Johnson, L. Moore and D. Ylvisaker. Minimax and maximin 
distance designs, \textit{Journal of Statistical Planning and 
Inference}, 26(2): 131-148, 1990.

\bibitem{maximin2}
T. Santner, B. Williams, and W. Notz. \textit{The Design and 
analysis of computer experiments}. Springer, 2nd edition, 
pp. 145-200, 2018.
 
\bibitem{entropy}
C.E. Shannon, A mathematical theory of communication. 
\textit{Bell System Technical Journal}, 27(3): 379-423, 1948.

\bibitem{max_entropy}
M.C. Shewry and H.P. Wynn. Maximum entropy sampling. 
\textit{Journal of Applied Statistics}, 14(2): 165-170, 1987.

%\bibitem{max_entrop2}
%J.R. Koehler and  A.B. Owen. Computer experiments. In  S. 
%Ghosh, and  C.R. Rao, \textit{Handbook of Statistics}, vol. 
%13, Elsevier Science, pages 261-308, 1996.

\bibitem{SE}
Y.G. Saab, Y.B. Rao. Combinatorial optimization by 
stochastic evolution IEEE Trans. \textit{Computer-Aided 
Design}, pp. 525-535, September 1991,

\bibitem{ESE}
R. Jin , W. Chen, and A. Sudjianto. An efficient 
algorithm for constructing optimal design of computer 
experiments. \textit{Journal of Statistical Planning and 
Inference}, 134(1): 268-287, 2005.

\bibitem{fp criterion}
M.D. Morris and  T.J. Mitchell. Exploratory Designs for 
Computational Experiments. \textit{Journal of statistical 
planning and inference}, 43: 381-402, 1995. 

\bibitem{discrepancy}
F.J. Hickernell. A  generalized  discrepancy  and  quadrature  
error  bound. \textit{Mathematics of Computation}, 67: 299-322, 
1998.  

\bibitem{Factorial}
D.C. Montgomery. \textit{Design and Analysis of Experiments}. 
Wiley, 9th edition, pp. 179-229, 2017.

%\bibitem{Full_Factorial1}
%P. Sahoo and T.Kr. Barman. ANN modelling of 
%fractal dimension in machining. In J.P. Davim,  
%\textit{Mechatronics and Manufacturing Engineering}, 
%Woodhead Publishing Reviews: Mechanical Engineering 
%Series, pages 159–226, 2012.

\bibitem{Full_Factorial2}
J. Antony. \textit{Design of Experiments for Engineers and 
Scientists}, chapter Full Factorial Designs. Elsevier 
Science, 2nd edition, pp. 63-85, 2014.

\bibitem{Fractional Factorial}
D.C. Montgomery. \textit{Design and Analysis of Experiments}. 
Wiley, 9th edition, pp. 351-384, 2017.

\bibitem{EASY}
EASY - The Evolutionary Algorithms SYstem Home, 2012. 
Retrieved from \url{http://velos0.ltt.mech.ntua.gr/EASY}
%------------------------------------------------------------

%-------------------On-line-----------------------------
\bibitem{LCPE}
M.K. Karakasis and K.C. Giannakoglou. On the use of 
metamodel-assisted, multi-objective evolutionary algorithms. 
\textit{Engineering Optimization}, 38(8):941–957, 2006.

\bibitem{TPS}
M.K. Karakasis and K.C. Giannakoglou. On the use of 
metamodel-assisted, multi-objective evolutionary 
algorithms. \textit{Engineering Optimization}, 38(8): 
941-957, 2006.
%------------------------------------------------------------

%------------------Metamodels--------------------------------
\bibitem{Kriging}
A. Keane, A. Forrester, and A. Sóbester. 
\textit{Engineering Design via Surrogate Modelling: A 
Practical Guide}. Wiley, 1st edition, 2008.
  
\bibitem{Matern}
C.E. Rasmussen and C.K.I. Williams. Gaussian Processes 
for Machine Learning, the MIT Press, 2006.
  
\bibitem{Kriging1}
G. Giangaspero, D. MacManus, and I. Goulos. Surrogate 
models for the prediction of the aerodynamic performance 
of exhaust systems. \textit{Aerospace Science and 
Technology}, 92: 77-90, 2019.

\bibitem{regression_model}
S.N. Lophaven, H.B. Nielsen, J. Søndergaard. \textit{DACE: 
a MatLab Kriging Toolbox}. Technical Report IMM-TR-2002-12, 
2002.

\bibitem{max_likelihood}
D.R. Jones. A Taxonomy of Global Optimization Methods Based on
Response Surfaces. \textit{Journal of Global Optimization}, 21(4): 
345–383, 2001.

\bibitem{BLUP}
J. Sacks, S.B. Schiller, and W.J. Welch. Designs for 
computer experiments.\textit{Technometrics}, 
31(1): 41-47, 1989.

\bibitem{COBYLA}
M.J.D. Powell. A direct search optimization method that 
models the objective and constraint functions by linear 
interpolation. In S. Gomez and J-P Hennart, 
\textit{Advances in Optimization and Numerical Analysis},
Kluwer Academic (Dordrecht), pp. 51-67, 1994.

\bibitem{noisy data}
A.E. Hoerl. (1962). Application of ridge analysis to regression 
problems. \textit{Chemical Engineering Progress} 58, 54-59, 1962.

\bibitem{PLS}
I. Helland. On structure of Partial Least Squares re-
gression. \textit{Communication in Statistics - Simulation 
and Computation} 17: 581–607, 1988.

\bibitem{KPLS}
M.A. Bouhlel, N. Bartoli, A. Otsmane, and J. Morlier. 
Improving kriging surrogates of high-dimensional design 
models by Partial Least Squares dimension reduction. 
\textit{Structural and Multidisciplinary Optimization}, 
53(5): 935-952, 2016.

\bibitem{power_iter}
C. Lanczos.  An iteration method for the solution of
the eigenvalue problem of linear differential and integral
operators. \textit{Journal of Research of the National 
Bureau of Standards}, 45(4): 255-282, 1950.

\bibitem{rotation_matrix}
R. Manne.  Analysis of two Partial-Least-Squares 
algorithms for multivariate calibration. 
\textit{Chemometrics and Intelligent Laboratory Systems}, 
2(1-3): 187-197, 1987. 

\bibitem{KPLSK}
M.A. Bouhlel, N. Bartoli, J. Morlier, and A. Otsmane. An 
improved approach for estimating the hyperparameters of 
the kriging model for high-dimensional problems through 
the partial least squares method. \textit{Mathematical 
Problems in Engineering}, 2016.

\bibitem{RBF}
M.J.D. Powell. \textit{The Theory of Radial Basis Function 
Approximation}. Oxford University Press, pp. 105-210, 1992.

\bibitem{RBF1}
M. N. Oqielat. Scattered data approximation using radial 
basis function with a cubic polynomial reproduction for 
modelling leaf surface. \textit{Journal of Taibah 
University for Science}, 12(3): 331-337, 2018.  

\bibitem{RBF2}
V. Bayona. An insight into RBF-FD approximations augmented 
with polynomials. \textit{Computers} \& 
\textit{Mathematics with Applications}, 77(9): 2337-2353, 
2019. 
%---------------------------------------------------------------

%-------------------------Numerical cases-----------------------
\bibitem{welded beam}
T. Ray and K. M. Liew. A Swarm Metaphor for 
Multiobjective Design Optimization. \textit{Engineering 
Optimization} 34: 141-153, 2002.

\bibitem{hypervolume indicator}
J.D. Knowles, D.W. Corne, and M. Fleischer. Bounded Archiving 
using the Lebesgue Measure. \textit{Congress on Evolutionary 
Computation}, 4: 2490–2497, 2003.

\bibitem{speed reducer}
S.S. Rao. \textit{Engineering Optimization: Theory and 
Practice}. Wiley, 5th ed., pp. 434-435, 2020.

\bibitem{Golinski}
J. Golinski. Optimal synthesis problems solved by means of 
nonlinear programming and random methods. \textit{Journal Of 
Mechanisms}, 5(3), 287-309, 1970.
%------------------------------------------------------------

%---------------------Airfoil--------------------------------
\bibitem{Spalart Allmaras}
P.R. Spalart and S.R. Allmaras. A One-Equation Turbulence Model 
for Aerodynamic Flows. \textit{30th Aerospace Sciences Meeting and 
Exhibit}, 1992.

\bibitem{PUMA}
K. Tsiakas. Development of shape parameterization techniques, a 
flow solver and its adjoint, for optimization on GPUs. 
Turbomachinery and external aero-dynamics applications. PhD thesis, 
Laboratory of Thermal Turbomachines, NTUA, Athens, 2019.

\bibitem{PUMA_GPU}
I.C. Kampolis, X.S Trompoukis, V.G. Asouti, K.C. Giannakoglou. 
CFD-based analysis and two-level aerodynamic optimization on 
graphics processing units. \textit{Computer Methods in Applied 
Mechanics and Engineering}, 199(9–12): 712–722, 2010.

\bibitem{Favre}
A.J.A Favre. Equations des gaz turbulents compressibles. 
\textit{Journal de Mecanique}, 4, 1965.

\bibitem{Flat plate SA}
G. Kalitzin, G. Medic, G. Iaccarino and P. Durbin. Near-wall 
behaviour of RANS turbulence models and implications for wall 
functions. \textit{Journal of Computational Physics}, 204: 265-291,
2005.

\bibitem{SA clarification}
S. Allmaras, F. Johnson, and P. Spalart. Modifications and 
clarifications for the implementation of the Spalart-Allmaras 
turbulence model. \textit{7th International Conference on 
Computational Fluid Dynamics}, 2012.

%------------------------------------------------------------

%-------------------Conclusion----------------------------------
%\bibitem{PCA}
%H. Abdi and L.J. Williams. Principal component analysis. 
%\textit{Wiley Interdisciplinary Reviews: Computational 
%Statistics}, 2(4): 433–459, 2010.

\bibitem{DEA}
F. Herrera, M. Lozano, and C. Moraga. Hierarchical 
distributed genetic algorithms. \textit{International 
Journal of Intelligent Systems}, 14(11): 1099–1121, 1999.

\bibitem{DEA1}
Y.-J. Gong, W.-N. Chen, Z.-H. Zhan, J. Zhang, Y. Li, Q. 
Zhang, and J.-J Li. Distributed evolutionary algorithms 
and their models: A survey of the state-of-the-art. 
\textit{Applied Soft Computing}, 34: 286–300, 2015.

\bibitem{DMAEA}
W. Annicchiarico. Metamodel-assisted distributed genetic 
algorithms applied to structural shape optimization 
problems. \textit{Engineering Optimization}, 39(7): 
757-772, 2007.

\bibitem{HEA}
J.F. Wang and J. Periaux and M. Sefrioui. Parallel 
evolutionary algorithms for optimization problems in 
aerospace engineering. \textit{Journal of Computational 
and Applied Mathematics}, 149(1): 155-169, 2002.

\bibitem{HDEA}
I.C. Kampolis and K.C. Giannakoglou. Distributed 
evolutionary algorithms with hierarchical evaluation. 
\textit{Engineering Optimization}, 41(11): 1037–1049, 
2009.

\bibitem{HDMAEA}
M. K. Karakasis, D. G Koubogiannis, and K. C Giannakoglou. 
Hierarchical distributed metamodel-assisted evolutionary 
algorithms in shape optimization. \textit{International 
Journal for Numerical Methods in Fluids}, 53(3): 455–469, 
2006.

%\bibitem{PEA}
%P. Adamidis. \textit{Parallel Evolutionary Algorithms: A 
%Review}, 1999.

\bibitem{MA}
N. Krasnogor and J. Smith. A tutorial for competent 
memetic algorithms: model, taxonomy, and design issues. 
\textit{IEEE Transactions on Evolutionary Computation},
9(5): 474–488, 2005.

\bibitem{MAMA}
C.A. Georgopoulou and K.C. Giannakoglou.
Memetic Algorithms. \textit{Springer Series}, 2009.

\bibitem{MAMA1}
C.A. Georgopoulou and K.C. Giannakoglou. A multi-objective 
metamodel-assisted memetic algorithm with strength-based 
local refinement. \textit{Engineering Optimization},
41(10): 909–923, 2009.

\bibitem{AEA}
E.O. Scott and K.A De Jong. Understanding Simple 
Asynchronous Evolutionary Algorithms. \textit{Proceedings 
of the 2015 ACM Conference on Foundations of Genetic 
Algorithms XIII}. FOGA ’15: Foundations of Genetic 
Algorithms XIII, 2015.

\bibitem{AEA1}
V.G. Asouti and K.C. Giannakoglou. Aerodynamic 
optimization using a parallel asynchronous evolutionary 
algorithm controlled by strongly interacting demes. 
\textit{Engineering Optimization}, 41(3): 241–257, 2009.

\bibitem{AMAEA}
V.G. Asouti, I.C. Kampolis, and K.C. Giannakoglou. A 
grid-enabled asynchronous metamodel-assisted evolutionary 
algorithm for aerodynamic optimization. \textit{Genetic
Programming and Evolvable Machines (SI:Parallel and 
Distributed Evolutionary Algorithms, Part One)}, 
10(3): 373–389, 2009.
%------------------------------------------------------------

%------------------Appendix--------------------------
 
\bibitem{wing_weight}
D.P. Raymer. Aircraft Design: A Conceptual Approach. AIAA 
Education Series, 6th edition, pp. 559-584, 2018. 

\bibitem{Evektor}
A. Pistek and R. Popela. The VUT 100/200 General 
Aviation Aircraft Family: Project and Realization. 
Proceedings of the Institution of Mechanical Engineers, 
Part G: \textit{Journal of Aerospace Engineering}, 221(2): 
193-197, 2007.

\bibitem{error_scores}
\textit{3.3. Metrics and scoring: quantifying the quality 
of prediction}s — scikit-learn 0.24.1 documentation. 
Retrieved January 17, 2021 from 
\url{https://scikit-learn.org/stable/modules/
model_evaluation.html#regression-metrics}.

\bibitem{pyDOE}
pyDOE: The experimental design package for python — pyDOE 
0.3.6 documentation. Retrieved February 2021 from 
\url{https://pythonhosted.org/pyDOE/index.html}.

\bibitem{scipy}
scipy.optimize.fmin\_cobyla — SciPy v1.6.1 Reference 
Guide. Retrieved 17 January, 2021 from 
\url{https://docs.scipy.org/doc/scipy/reference/generated/
scipy.optimize.fmin_cobyla.html}.
